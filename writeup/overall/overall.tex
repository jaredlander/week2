
\documentclass{article}
\usepackage{knitr}
\newcommand{\SweaveOpts}[1]{}  % do not interfere with LaTeX
\newcommand{\SweaveInput}[1]{} % because they are not real TeX commands
\newcommand{\Sexpr}[1]{}       % will only be parsed by R



% get required packages
\usepackage{graphicx}   % for graphics
\usepackage{mathtools}  % for math formulas
\usepackage{amssymb}    % for more math styles
\usepackage{amsfonts}    % for more math styles
%\usepackage{amsmath}
%\usepackage[retainorgcmds]{IEEEtrantools}   % advanced equation alignment
\usepackage{makeidx}    % for index generation
\usepackage{showidx}

% we are using primarily png graphics
\DeclareGraphicsExtensions{.png,.pdf}

% make an index
\makeindex

% Title of book
\newcommand{\myTitle}{Generalizing from Purposive Surveys\\ How large a Sample is Needed}

% This puts the chapter name at the top of the page
\pagestyle{headings}





\title{\myTitle}
\author{Richard Garfield\\ Columbia University School of Nursing \and Jared P. Lander\\ JP Lander Consulting}

%%%%%%%%%%%%%%%%%%%%%%%%%%
% the main document
%%%%%%%%%%%%%%%%%%%%%%%%%%


\begin{document}
% !Rnw weave = knitr


% The Data
\section{Analyzing All Data}
\label{sec:overall}
Here we analyze all of the data.

First we load the data and view a portion of it. Some more details.

These are the necessary packages.
\begin{knitrout}
\definecolor{shadecolor}{rgb}{1, 1, 1}\color{fgcolor}\begin{kframe}
\begin{alltt}
\hlfunctioncall{require}(useful)
\hlfunctioncall{require}(plyr)
\hlfunctioncall{require}(ggplot2)
\end{alltt}
\end{kframe}
\end{knitrout}


\begin{knitrout}
\definecolor{shadecolor}{rgb}{1, 1, 1}\color{fgcolor}\begin{kframe}
\begin{alltt}
\hlfunctioncall{load}(\hlstring{"../../data/pakistan/pak.rdata"})
\hlfunctioncall{source}(\hlstring{"../../R/distFuncs.r"})
\hlfunctioncall{corner}(pak, c = 15)
\end{alltt}
\begin{verbatim}
  New_ID Age  Sex     Date Province District Tehsil
1   1288  26 Male 29082010      KPK  Shangla Besham
2   1290  30 Male 29082010      KPK  Shangla Besham
3   1370  54 Male 28082010      KPK  Shangla Besham
4   1372  53 Male 28082010      KPK  Shangla Besham
5   1371  64 Male 28082010      KPK  Shangla Besham
         Village Latitude Longitude Total Urban Rural
1 abaseen colony    34.94     72.88  90.6     -  90.6
2 abaseen colony    34.94     72.88  90.6     -  90.6
3 abaseen colony    34.94     72.88  90.6     -  90.6
4 abaseen colony    34.94     72.88  90.6     -  90.6
5 abaseen colony    34.94     72.88  90.6     -  90.6
                                Accommodation
1 Collective centers (school/Public building)
2                                 Host family
3          On the site of the house (Damaged)
4          On the site of the house (Damaged)
5          On the site of the house (Damaged)
  StagnantWater
1           Few
2           Few
3           Few
4          None
5          None
\end{verbatim}
\end{kframe}
\end{knitrout}


Now we build a distribution for all the data and visualize it in Figure~\ref{fig:overallDist} with the code here:.
\begin{knitrout}
\definecolor{shadecolor}{rgb}{1, 1, 1}\color{fgcolor}\begin{kframe}
\begin{alltt}
ricePerc <- \hlfunctioncall{build.dist}(data = pak, lhs = \hlstring{"New_ID"}, group = \hlstring{"Province"}, 
    question = \hlstring{"RiceLost"})
ricePerc$Size <- \hlstring{"All"}
\hlfunctioncall{ggplot}(ricePerc, \hlfunctioncall{aes}(x = RiceLost, y = Percent)) + \hlfunctioncall{geom_bar}(stat = \hlstring{"identity"}) + 
    \hlfunctioncall{facet_wrap}(~Province) + \hlfunctioncall{opts}(axis.text.x = \hlfunctioncall{theme_text}(angle = 90))
\end{alltt}
\end{kframe}\begin{figure}[!hbtp]


{\centering \includegraphics[width=.8\linewidth]{figures/overallDist} 

}

\caption[Graphical view of the distribution of responses for all the data]{Graphical view of the distribution of responses for all the data.\label{fig:overallDist}}
\end{figure}


\end{knitrout}



In Section~\ref{sec:smallerDist} we analyze the distribution of responses for samples of fewer Tehsils.
\end{document}
