% !Rnw weave = knitr
% Writeup of results
% master file

%%%%%%%%%%%%%%%%%%%%%%%%%%
% preamble
%%%%%%%%%%%%%%%%%%%%%%%%%%

% define document class
\documentclass{article}\usepackage{knitr}

% get required packages
\usepackage{graphicx}   % for graphics
\usepackage{mathtools}  % for math formulas
\usepackage{amssymb}    % for more math styles
\usepackage{amsfonts}    % for more math styles
%\usepackage{amsmath}
%\usepackage[retainorgcmds]{IEEEtrantools}   % advanced equation alignment
\usepackage{makeidx}    % for index generation
\usepackage{showidx}

% we are using primarily png graphics
\DeclareGraphicsExtensions{.png,.pdf}

% make an index
\makeindex

% Title of book
\newcommand{\myTitle}{Generalizing from Purposive Surveys\\ How large a Sample is Needed}

% This puts the chapter name at the top of the page
\pagestyle{headings}





\title{\myTitle}
\author{Richard Garfield\\ Columbia University School of Nursing \and Jared P. Lander\\ JP Lander Consulting}

%%%%%%%%%%%%%%%%%%%%%%%%%%
% the main document
%%%%%%%%%%%%%%%%%%%%%%%%%%
\begin{document}
\maketitle      % create the title page
\tableofcontents    % table of contents

% !Rnw weave = knitr






\section{Distribution Functions}
\label{sec:DistributionFunctions}
These are the functions used to calculate the distribution of each answer. They are general and should work with any question.

\begin{knitrout}
\definecolor{shadecolor}{rgb}{1, 1, 1}\color{fgcolor}\begin{kframe}
\begin{alltt}
\hlfunctioncall{getwd}()
\end{alltt}
\begin{verbatim}
[1] "C:/Users/Jared/week2/writeup/distFuncs"
\end{verbatim}
\end{kframe}
\end{knitrout}


\begin{knitrout}
\definecolor{shadecolor}{rgb}{1, 1, 1}\color{fgcolor}\begin{kframe}
\begin{alltt}
\hlcomment{# Distribution functions}
\hlfunctioncall{require}(useful)
\hlfunctioncall{require}(plyr)
\hlcomment{## builds the distribution for a given question}
build.dist <- \hlfunctioncall{function}(data, lhs, group, question)
\{
    theFormula <- \hlfunctioncall{build.formula}(lhs = lhs, rhs = \hlfunctioncall{c}(group, 
        question))
    agg <- \hlfunctioncall{aggregate}(theFormula, data, length)
    agg <- \hlfunctioncall{ddply}(agg, .variables = group, .fun = \hlfunctioncall{function}(x)
    \{
        x$Percent <- x[[lhs]]/\hlfunctioncall{sum}(x[[lhs]])
        \hlfunctioncall{return}(x)
    \})
    agg
\}
\hlcomment{## get random tehsils from a province}
village.list <- \hlfunctioncall{function}(x, num = 5, unit = \hlstring{"Tehsil"})
\{
\hlcomment{    # get list of units}
    units <- \hlfunctioncall{unique}(x[, unit])
    
\hlcomment{    # sample num of those without replacement}
    keepers <- \hlfunctioncall{sample}(x = units, size = \hlfunctioncall{min}(num, \hlfunctioncall{length}(units)), 
        replace = FALSE)
    
    \hlfunctioncall{return}(\hlfunctioncall{as.character}(keepers))
\}
\hlcomment{# function to make names of dist's better}
change.names <- \hlfunctioncall{function}(names, include = names, prefix = \hlstring{""})
\{
    theOnes <- \hlfunctioncall{which}(!names %in% include)
    names[theOnes] <- \hlfunctioncall{sprintf}(\hlstring{"%s.%s"}, prefix, names[theOnes])
    \hlfunctioncall{return}(names)
\}
\hlcomment{## function to impute missing}
impute.col <- \hlfunctioncall{function}(col, value = 0)
\{
    col[\hlfunctioncall{is.na}(col)] <- value
    \hlfunctioncall{return}(col)
\}
\hlcomment{## this compares two distributions and computes an MSE}
compare.dist <- \hlfunctioncall{function}(full, partial, compare = \hlstring{"Percent"}, 
    by = \hlfunctioncall{intersect}(\hlfunctioncall{names}(full), \hlfunctioncall{names}(partial)))
\{
\hlcomment{    # prepend Pull onto certain names in full}
    \hlfunctioncall{names}(full) <- \hlfunctioncall{change.names}(names = \hlfunctioncall{names}(full), include = by, 
        prefix = \hlstring{"Full"})
    
\hlcomment{    # prepend Partial onto certain names in full}
    \hlfunctioncall{names}(partial) <- \hlfunctioncall{change.names}(names = \hlfunctioncall{names}(partial), 
        include = by, prefix = \hlstring{"Partial"})
    
    full.compare <- \hlfunctioncall{sprintf}(\hlstring{"Full.%s"}, compare)
    partial.compare <- \hlfunctioncall{sprintf}(\hlstring{"Partial.%s"}, compare)
    
\hlcomment{    # join the two together}
    both <- \hlfunctioncall{join}(x = full, y = partial, by = by, type = \hlstring{"left"})
    
    \hlfunctioncall{rm}(full, partial)
    
\hlcomment{    ## fill in any NA's with zero}
    both[[full.compare]] <- \hlfunctioncall{impute.col}(col = both[[full.compare]], 
        value = 0)
    both[[partial.compare]] <- \hlfunctioncall{impute.col}(col = both[[partial.compare]], 
        value = 0)
    
    both$.Diff <- both[[full.compare]] - both[[partial.compare]]
    
    both$.MSE <- \hlfunctioncall{mean}(both$.Diff^2)
    
\hlcomment{    # attr(x=both, which='MSE') <- mean(both$.Diff^2)}
    
\hlcomment{    # aggregate(build.formula(lhs='.Diff', rhs=}
    
    \hlfunctioncall{return}(both)
\}
\end{alltt}
\end{kframe}
\end{knitrout}


% <<functions>>=
% # Distribution functions
% require(useful)
% ## builds the distribution for a given question
% build.dist <- function(data, lhs, group, question)
% {
%     theFormula <- build.formula(lhs=lhs, rhs=c(group, question))
%     agg <- aggregate(theFormula, data, length)
%     agg <- ddply(agg, .variables=group, .fun=function(x){ x$Percent <- x[[lhs]] / sum(x[[lhs]]); return(x) })
%     agg
% }
% 
% 
% ## get random Tehsils from a province
% village.list <- function(x, num=5, unit="Tehsil")
% {
%     # get list of units
%     units <- unique(x[, unit])
%     
%     # sample num of those without replacement
%     keepers <- sample(x=units, size=min(num, length(units)), replace=FALSE)
%     
%     return(as.character(keepers))
% }
% 
% 
% # function to make names of dist's better
% change.names <- function(names, include=names, prefix="")
% {
%     theOnes <- which(!names %in% include)
%     names[theOnes] <- sprintf("%s.%s", prefix, names[theOnes])
%     return(names)
% }
% 
% ## function to impute missing
% impute.col <- function(col, value=0)
% {
%     col[is.na(col)] <- value
%     return(col)
% }
% 
% ## this compares two distributions and computes an MSE
% compare.dist <- function(full, partial, compare="Percent", by=intersect(names(full), names(partial)))
% {
%     # prepend Pull onto certain names in full
%     names(full) <- change.names(names=names(full), include=by, prefix="Full")
%     
%     # prepend Partial onto certain names in full
%     names(partial) <- change.names(names=names(partial), include=by, prefix="Partial")
%     
%     full.compare <- sprintf("Full.%s", compare)
%     partial.compare <- sprintf("Partial.%s", compare)
%     
%     # join the two together
%     both <- join(x=full, y=partial, by=by, type="left")
%     
%     rm(full, partial)
%     
%     ## fill in any NA's with zero
%     both[[full.compare]] <- impute.col(col=both[[full.compare]], value=0)
%     both[[partial.compare]] <- impute.col(col=both[[partial.compare]], value=0)
%     
%     both$.Diff <- both[[full.compare]] - both[[partial.compare]]
%     
%     both$.MSE <- mean(both$.Diff^2)
%     
%     #attr(x=both, which="MSE") <- mean(both$.Diff^2)
%     
%     #aggregate(build.formula(lhs=".Diff", rhs=
%     
%     return(both)
% }
% @

% !Rnw weave = knitr
% Initial stuff
\section{Initial Stuff}
\label{sec:initial}
The data is as described in Section~\ref{sec:thedata}.

We examined the answer to the question``What percentage of rice crops were lost due to the flood?''  We then randomly chose five Tehsils from each province, then 10, then 15 and performed the same analysis on the reduced data.

In situations where a province has fewer than five, 10 or 15 Tehsils sampled, all were used.

% !Rnw weave = knitr



% The Data



\section{The Data}
\label{sec:thedata}


The data were collected following the floods in Pakistan in 2010. Small Changes.

It surveyed affected villages in GB, KPK, Punjab and Sindh.

% make code to plot Vill distribution but don't run or show it.
% <<VillDist,include=FALSE,eval=FALSE>>=
% ggplot(vills, aes(x=Tehsil)) + geom_bar(aes(y=Village), stat="identity") + opts(axis.text.x=theme_text(angle=270, hjust=0)) + facet_wrap(~Province, scales="free_x")
% @

The distribution of villages within Tehsils within Provinces is seen in Figure~\ref{fig:VillDist}.

\begin{knitrout}
\definecolor{shadecolor}{rgb}{1, 1, 1}\color{fgcolor}\begin{kframe}
\begin{alltt}
\hlfunctioncall{ggplot}(vills, \hlfunctioncall{aes}(x = Tehsil)) + \hlfunctioncall{geom_bar}(\hlfunctioncall{aes}(y = Village), 
    stat = \hlstring{"identity"}) + \hlfunctioncall{opts}(axis.text.x = \hlfunctioncall{theme_text}(angle = 270, 
    hjust = 0)) + \hlfunctioncall{facet_wrap}(~Province, scales = \hlstring{"free_x"})
\end{alltt}
\end{kframe}\begin{figure}[!hbtp]


{\centering \includegraphics[width=.9\linewidth]{figures/VillDist} 

}

\caption[Distribution of villages within Tehsils]{Distribution of villages within Tehsils within the four Provinces.\label{fig:VillDist}}
\end{figure}


\end{knitrout}

The analysis begins in Section~\ref{sec:overall}.

% !Rnw weave = knitr


% The Data
\section{Analyzing All Data}
\label{sec:overall}
Here we analyze all of the data.

First we load the data and view a portion of it. Some more details.

These are the necessary packages.
\begin{knitrout}
\definecolor{shadecolor}{rgb}{1, 1, 1}\color{fgcolor}\begin{kframe}
\begin{alltt}
\hlfunctioncall{require}(useful)
\hlfunctioncall{require}(plyr)
\hlfunctioncall{require}(ggplot2)
\end{alltt}
\end{kframe}
\end{knitrout}


\begin{knitrout}
\definecolor{shadecolor}{rgb}{1, 1, 1}\color{fgcolor}\begin{kframe}
\begin{alltt}
\hlfunctioncall{load}(\hlstring{"../../data/pakistan/pak.rdata"})
\hlfunctioncall{source}(\hlstring{"../../R/distFuncs.r"})
\hlfunctioncall{corner}(pak, c = 15)
\end{alltt}
\begin{verbatim}
  New_ID Age  Sex     Date Province District Tehsil
1   1288  26 Male 29082010      KPK  Shangla Besham
2   1290  30 Male 29082010      KPK  Shangla Besham
3   1370  54 Male 28082010      KPK  Shangla Besham
4   1372  53 Male 28082010      KPK  Shangla Besham
5   1371  64 Male 28082010      KPK  Shangla Besham
         Village Latitude Longitude Total Urban Rural
1 abaseen colony    34.94     72.88  90.6     -  90.6
2 abaseen colony    34.94     72.88  90.6     -  90.6
3 abaseen colony    34.94     72.88  90.6     -  90.6
4 abaseen colony    34.94     72.88  90.6     -  90.6
5 abaseen colony    34.94     72.88  90.6     -  90.6
                                Accommodation
1 Collective centers (school/Public building)
2                                 Host family
3          On the site of the house (Damaged)
4          On the site of the house (Damaged)
5          On the site of the house (Damaged)
  StagnantWater
1           Few
2           Few
3           Few
4          None
5          None
\end{verbatim}
\end{kframe}
\end{knitrout}


Now we build a distribution for all the data and visualize it in Figure~\ref{fig:overallDist} with the code here:.
\begin{knitrout}
\definecolor{shadecolor}{rgb}{1, 1, 1}\color{fgcolor}\begin{kframe}
\begin{alltt}
ricePerc <- \hlfunctioncall{build.dist}(data = pak, lhs = \hlstring{"New_ID"}, group = \hlstring{"Province"}, 
    question = \hlstring{"RiceLost"})
ricePerc$Size <- \hlstring{"All"}
\hlfunctioncall{ggplot}(ricePerc, \hlfunctioncall{aes}(x = RiceLost, y = Percent)) + \hlfunctioncall{geom_bar}(stat = \hlstring{"identity"}) + 
    \hlfunctioncall{facet_wrap}(~Province) + \hlfunctioncall{opts}(axis.text.x = \hlfunctioncall{theme_text}(angle = 90))
\end{alltt}
\end{kframe}\begin{figure}[!hbtp]


{\centering \includegraphics[width=.8\linewidth]{figures/overallDist} 

}

\caption[Graphical view of the distribution of responses for all the data]{Graphical view of the distribution of responses for all the data.\label{fig:overallDist}}
\end{figure}


\end{knitrout}



In Section~\ref{sec:smallerDist} we analyze the distribution of responses for samples of fewer Tehsils.

% <<smaller-dist-section,child="smallerDist/smallerDist.Rnw">>=
% @

\cleardoublepage
\addcontentsline{toc}{section}{\numberline{}List of Figures}
\listoffigures
%\cleardoublepage
%\listoftables
%\cleardoublepage
\printindex
\end{document}
