
\documentclass{article}
\usepackage{knitr}
\newcommand{\SweaveOpts}[1]{}  % do not interfere with LaTeX
\newcommand{\SweaveInput}[1]{} % because they are not real TeX commands
\newcommand{\Sexpr}[1]{}       % will only be parsed by R



% get required packages
\usepackage{graphicx}   % for graphics
\usepackage{mathtools}  % for math formulas
\usepackage{amssymb}    % for more math styles
\usepackage{amsfonts}    % for more math styles
%\usepackage{amsmath}
%\usepackage[retainorgcmds]{IEEEtrantools}   % advanced equation alignment
\usepackage{makeidx}    % for index generation
\usepackage{showidx}

% we are using primarily png graphics
\DeclareGraphicsExtensions{.png,.pdf}

% make an index
\makeindex

% Title of book
\newcommand{\myTitle}{Generalizing from Purposive Surveys\\ How large a Sample is Needed}

% This puts the chapter name at the top of the page
\pagestyle{headings}





\title{\myTitle}
\author{Richard Garfield\\ Columbia University School of Nursing \and Jared P. Lander\\ JP Lander Consulting}

%%%%%%%%%%%%%%%%%%%%%%%%%%
% the main document
%%%%%%%%%%%%%%%%%%%%%%%%%%


\begin{document}
% !Rnw weave = knitr






\section{Distribution Functions}
\label{sec:DistributionFunctions}
These are the functions used to calculate the distribution of each answer. They are general and should work with any question.

\begin{knitrout}
\definecolor{shadecolor}{rgb}{1, 1, 1}\color{fgcolor}\begin{kframe}
\begin{alltt}
\hlfunctioncall{getwd}()
\end{alltt}
\begin{verbatim}
[1] "C:/Users/Jared/week2/writeup/distFuncs"
\end{verbatim}
\end{kframe}
\end{knitrout}


\begin{knitrout}
\definecolor{shadecolor}{rgb}{1, 1, 1}\color{fgcolor}\begin{kframe}
\begin{alltt}
\hlcomment{# Distribution functions}
\hlfunctioncall{require}(useful)
\hlcomment{## builds the distribution for a given question}
build.dist <- \hlfunctioncall{function}(data, lhs, group, question)
\{
    theFormula <- \hlfunctioncall{build.formula}(lhs = lhs, rhs = \hlfunctioncall{c}(group, 
        question))
    agg <- \hlfunctioncall{aggregate}(theFormula, data, length)
    agg <- \hlfunctioncall{ddply}(agg, .variables = group, .fun = \hlfunctioncall{function}(x)
    \{
        x$Percent <- x[[lhs]]/\hlfunctioncall{sum}(x[[lhs]])
        \hlfunctioncall{return}(x)
    \})
    agg
\}
\hlcomment{## get random tehsils from a province}
village.list <- \hlfunctioncall{function}(x, num = 5, unit = \hlstring{"Tehsil"})
\{
\hlcomment{    # get list of units}
    units <- \hlfunctioncall{unique}(x[, unit])
    
\hlcomment{    # sample num of those without replacement}
    keepers <- \hlfunctioncall{sample}(x = units, size = \hlfunctioncall{min}(num, \hlfunctioncall{length}(units)), 
        replace = FALSE)
    
    \hlfunctioncall{return}(\hlfunctioncall{as.character}(keepers))
\}
\hlcomment{# function to make names of dist's better}
change.names <- \hlfunctioncall{function}(names, include = names, prefix = \hlstring{""})
\{
    theOnes <- \hlfunctioncall{which}(!names %in% include)
    names[theOnes] <- \hlfunctioncall{sprintf}(\hlstring{"%s.%s"}, prefix, names[theOnes])
    \hlfunctioncall{return}(names)
\}
\hlcomment{## function to impute missing}
impute.col <- \hlfunctioncall{function}(col, value = 0)
\{
    col[\hlfunctioncall{is.na}(col)] <- value
    \hlfunctioncall{return}(col)
\}
\hlcomment{## this compares two distributions and computes an MSE}
compare.dist <- \hlfunctioncall{function}(full, partial, compare = \hlstring{"Percent"}, 
    by = \hlfunctioncall{intersect}(\hlfunctioncall{names}(full), \hlfunctioncall{names}(partial)))
\{
\hlcomment{    # prepend Pull onto certain names in full}
    \hlfunctioncall{names}(full) <- \hlfunctioncall{change.names}(names = \hlfunctioncall{names}(full), include = by, 
        prefix = \hlstring{"Full"})
    
\hlcomment{    # prepend Partial onto certain names in full}
    \hlfunctioncall{names}(partial) <- \hlfunctioncall{change.names}(names = \hlfunctioncall{names}(partial), 
        include = by, prefix = \hlstring{"Partial"})
    
    full.compare <- \hlfunctioncall{sprintf}(\hlstring{"Full.%s"}, compare)
    partial.compare <- \hlfunctioncall{sprintf}(\hlstring{"Partial.%s"}, compare)
    
\hlcomment{    # join the two together}
    both <- \hlfunctioncall{join}(x = full, y = partial, by = by, type = \hlstring{"left"})
    
    \hlfunctioncall{rm}(full, partial)
    
\hlcomment{    ## fill in any NA's with zero}
    both[[full.compare]] <- \hlfunctioncall{impute.col}(col = both[[full.compare]], 
        value = 0)
    both[[partial.compare]] <- \hlfunctioncall{impute.col}(col = both[[partial.compare]], 
        value = 0)
    
    both$.Diff <- both[[full.compare]] - both[[partial.compare]]
    
    both$.MSE <- \hlfunctioncall{mean}(both$.Diff^2)
    
\hlcomment{    # attr(x=both, which='MSE') <- mean(both$.Diff^2)}
    
\hlcomment{    # aggregate(build.formula(lhs='.Diff', rhs=}
    
    \hlfunctioncall{return}(both)
\}
\end{alltt}
\end{kframe}
\end{knitrout}


% <<functions>>=
% # Distribution functions
% require(useful)
% ## builds the distribution for a given question
% build.dist <- function(data, lhs, group, question)
% {
%     theFormula <- build.formula(lhs=lhs, rhs=c(group, question))
%     agg <- aggregate(theFormula, data, length)
%     agg <- ddply(agg, .variables=group, .fun=function(x){ x$Percent <- x[[lhs]] / sum(x[[lhs]]); return(x) })
%     agg
% }
% 
% 
% ## get random Tehsils from a province
% village.list <- function(x, num=5, unit="Tehsil")
% {
%     # get list of units
%     units <- unique(x[, unit])
%     
%     # sample num of those without replacement
%     keepers <- sample(x=units, size=min(num, length(units)), replace=FALSE)
%     
%     return(as.character(keepers))
% }
% 
% 
% # function to make names of dist's better
% change.names <- function(names, include=names, prefix="")
% {
%     theOnes <- which(!names %in% include)
%     names[theOnes] <- sprintf("%s.%s", prefix, names[theOnes])
%     return(names)
% }
% 
% ## function to impute missing
% impute.col <- function(col, value=0)
% {
%     col[is.na(col)] <- value
%     return(col)
% }
% 
% ## this compares two distributions and computes an MSE
% compare.dist <- function(full, partial, compare="Percent", by=intersect(names(full), names(partial)))
% {
%     # prepend Pull onto certain names in full
%     names(full) <- change.names(names=names(full), include=by, prefix="Full")
%     
%     # prepend Partial onto certain names in full
%     names(partial) <- change.names(names=names(partial), include=by, prefix="Partial")
%     
%     full.compare <- sprintf("Full.%s", compare)
%     partial.compare <- sprintf("Partial.%s", compare)
%     
%     # join the two together
%     both <- join(x=full, y=partial, by=by, type="left")
%     
%     rm(full, partial)
%     
%     ## fill in any NA's with zero
%     both[[full.compare]] <- impute.col(col=both[[full.compare]], value=0)
%     both[[partial.compare]] <- impute.col(col=both[[partial.compare]], value=0)
%     
%     both$.Diff <- both[[full.compare]] - both[[partial.compare]]
%     
%     both$.MSE <- mean(both$.Diff^2)
%     
%     #attr(x=both, which="MSE") <- mean(both$.Diff^2)
%     
%     #aggregate(build.formula(lhs=".Diff", rhs=
%     
%     return(both)
% }
% @
\end{document}
