\section{Distribution Functions}
\label{sec:DistributionFunctions}
These are the functions used to calculate the distribution of each answer. They are general and should workd with any question.

\begin{knitrout}
\definecolor{shadecolor}{rgb}{1, 1, 1}\color{fgcolor}\begin{kframe}
\begin{alltt}
\hlcomment{# Distribution functions}
\hlfunctioncall{require}(useful)
\hlcomment{## builds the distribution for a given question}
build.dist <- \hlfunctioncall{function}(data, lhs, group, question) \{
    theFormula <- \hlfunctioncall{build.formula}(lhs = lhs, rhs = \hlfunctioncall{c}(group, question))
    agg <- \hlfunctioncall{aggregate}(theFormula, data, length)
    agg <- \hlfunctioncall{ddply}(agg, .variables = group, .fun = \hlfunctioncall{function}(x) \{
        x$Percent <- x[[lhs]]/\hlfunctioncall{sum}(x[[lhs]])
        \hlfunctioncall{return}(x)
    \})
    agg
\}


\hlcomment{## get random Tehsils from a province}
village.list <- \hlfunctioncall{function}(x, num = 5, unit = \hlstring{"Tehsil"}) \{
    \hlcomment{# get list of units}
    units <- \hlfunctioncall{unique}(x[, unit])
    
    \hlcomment{# sample num of those without replacement}
    keepers <- \hlfunctioncall{sample}(x = units, size = \hlfunctioncall{min}(num, \hlfunctioncall{length}(units)), replace = FALSE)
    
    \hlfunctioncall{return}(\hlfunctioncall{as.character}(keepers))
\}


\hlcomment{# function to make names of dist's better}
change.names <- \hlfunctioncall{function}(names, include = names, prefix = \hlstring{""}) \{
    theOnes <- \hlfunctioncall{which}(!names %in% include)
    names[theOnes] <- \hlfunctioncall{sprintf}(\hlstring{"%s.%s"}, prefix, names[theOnes])
    \hlfunctioncall{return}(names)
\}

\hlcomment{## function to impute missing}
impute.col <- \hlfunctioncall{function}(col, value = 0) \{
    col[\hlfunctioncall{is.na}(col)] <- value
    \hlfunctioncall{return}(col)
\}

\hlcomment{## this compares two distributions and computes an MSE}
compare.dist <- \hlfunctioncall{function}(full, partial, compare = \hlstring{"Percent"}, by = \hlfunctioncall{intersect}(\hlfunctioncall{names}(full), 
    \hlfunctioncall{names}(partial))) \{
    \hlcomment{# prepend Pull onto certain names in full}
    \hlfunctioncall{names}(full) <- \hlfunctioncall{change.names}(names = \hlfunctioncall{names}(full), include = by, prefix = \hlstring{"Full"})
    
    \hlcomment{# prepend Partial onto certain names in full}
    \hlfunctioncall{names}(partial) <- \hlfunctioncall{change.names}(names = \hlfunctioncall{names}(partial), include = by, prefix = \hlstring{"Partial"})
    
    full.compare <- \hlfunctioncall{sprintf}(\hlstring{"Full.%s"}, compare)
    partial.compare <- \hlfunctioncall{sprintf}(\hlstring{"Partial.%s"}, compare)
    
    \hlcomment{# join the two together}
    both <- \hlfunctioncall{join}(x = full, y = partial, by = by, type = \hlstring{"left"})
    
    \hlfunctioncall{rm}(full, partial)
    
    \hlcomment{## fill in any NA's with zero}
    both[[full.compare]] <- \hlfunctioncall{impute.col}(col = both[[full.compare]], value = 0)
    both[[partial.compare]] <- \hlfunctioncall{impute.col}(col = both[[partial.compare]], value = 0)
    
    both$.Diff <- both[[full.compare]] - both[[partial.compare]]
    
    both$.MSE <- \hlfunctioncall{mean}(both$.Diff^2)
    
    \hlcomment{# \hlfunctioncall{attr}(x=both, which=\hlstring{'MSE'}) <- \hlfunctioncall{mean}(both$.Diff^2)}
    
    \hlcomment{# \hlfunctioncall{aggregate}(\hlfunctioncall{build.formula}(lhs=\hlstring{'.Diff'}, rhs=}
    
    \hlfunctioncall{return}(both)
\}
\end{alltt}
\end{kframe}
\end{knitrout}

